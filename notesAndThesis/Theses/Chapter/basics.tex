

\graphicspath{{Chapter/Figs/basics/}}
\chapter{Foundations and Technologies}
\label{chap:basics}

\section{Discrete Curvature}
Discrete curvature applies the concept of curvature from a continuous curve to a discrete curve called a polyline.

\vskip 0.2in

\picWrap{Polyline over $P_0$-$P_6$}{polyline}{0.3}{R}
A polyline is a series of line segments and is determined by a sequence of points $(P_0,...,P_n)$ $n \in \mathbb{N}$ where each line segment connecting a pair of adjacent points $[P_i,P_{i+1}]$ $i \in \mathbb{N}_{\le n}$ forms a vertex in the polyline.

\vskip 0.2in

In the continuum[source:wiki] "the curvature at a point of a differentiable curve is the curvature of its osculating circle"  which more formally can be defined in terms of the unit tangent $\vec T$ and the arc length $s$: \citep{CurvatureDefinition} \citep{Aup91}

$$\kappa  = \left\| {\frac{{d\,\vec T}}{{ds}}} \right\|$$

This definition however cannot be used directly to determine the curvature of points in a polyline, given its non-continuous nature. All straight segments would have a curvature of $0$ while the curvature in the edges would diverge to infinity. A new definition must be used to determine the curvature of a series line segments, which in turn can be used to approximate this series. A different definition can be derived from the quotient of the circular angle and the arc length:

$$\kappa  =  {\frac{{d\,\varphi}}{{ds}}}$$

Using this idea we can define the curvature from a point $A$, a heading  $\vec h$ in that point and a point $B$ as the reciprocal of the radius of the circle passing though $A$ and $B$ and being tangent to  $\vec h$ in $A$.

\pic{Points $A$ with heading $\vec h$ and $B$ in circle}{discretecurvature}{0.5}

Thus we can calculate the reciprocal of the radius of this circle as follows:

Since $\vec h$ is tangent $\gamma=90^{\circ} - \alpha$ and $180^{\circ}=2\gamma+\beta$ thus $\beta = 2 \alpha$

\picWrap{Example Curvature approximating a polyline in $A$}{curvatureExample}{0.25}{L}

The length of the secant $s:=|\vec{AB}|$ can be calculated as
$s = 2r \cdot  sin(\frac{\beta}{2})$
$\Leftrightarrow$
$\frac{1}{r} = \frac{2sin(\alpha)}{s}$

Using this method we can calculate the average curvature of the curve that is tangent in $A$ to $\vec h$ and passing though $B$, which approximates the polyline connecting these points. The heading  $\vec h$ can also be derived using the next point after $A$ leading to $B$.
Doing this for different distant points $B$ on a polyline gives us a suitable approximation for the course of a polyline starting from point $A$. Of course this neglects the shape of the polyline completely, which fails to detect S-curves between point $A$ and $B$, this however imposes no problem if we choose a fairly small distance between point $A$ and $B$ such that the variance of the curvature for intermediate points is non-significant.