\chapter{Conclusion}
\label{chap:end}

\section{Summary}
Fasse nochmal alle Ergebnisse der Arbeit zusammen.
\todo{0.5p}

\section{Future Work}
\subsection{SLAM}
While the \ac{slam} provides accurate information about the landmarks in a local environment around the driver, the information provided is still noisy and drifts over time. Since the detection is not perfect, the error on the position of detected landmarks accumulates and causes the estimated position to drift from the actual position. Another problem is the double detection of cones when seen from a pass close by, and later drive-through. These problems could be improved upon by exploring extensions to the currently used FastSLAM \cite{FastSLAM2002} as well as using different \ac{slam} algorithms entirely such as EKF SLAMs \cite{EKFSLAM1986} or GraphSLAM \cite{graphSLAM2006}. As improvements to the input data have a positive effect along the rest of the pipeline that follows these improvements could contribute a big part to overall system improvement.
\subsection{Classical Algorithm}
As seen in the evaluation, the classical algorithm can only be applied to find the centerline to the cones in a complete round, therefore it cannot be used in the first round to drive along the track in the first place. Changing the algorithm in a way that it can handle incomplete (and possible noisy new) data would make the classical algorithm usable for driving the first round as well. \\
\\Also, when used in the first round, the estimated position of the car, can be factored in to determine the importance of cones, such that potential double detections of distant track parts can be ignored this way. Furthermore, the orientation of the cones relative to the car, being on the left or right side of it, can be used to verify the color detection of the cones, as there is a correlation between the color of cones and the position relative to the car, given that the car has not left the track.
\subsection{Machine Learning Algorithm}
As  only original unaugmented data was used, one simple way to improve on the machine learning approach is to augment the data using mirroring and rotation. Another factor that can be used would be the deviation from the centerline, as currently only training samples are used where the car is perfectly centered in the track. Such deviations are handled implicitly instead of handling deviations explicitly. One way of handling those would be to augment the input data by translating them along the x-axis and altering the expected curvatures to account for the additional steering that needs to take place to return the car back to the centerline. Another possibility is to add the deviation from the centerline as an additional output parameter of the network. This way the network learns to estimate the deviation along the future trajectory of the course.
\subsection{Other Improvements}
Another improvement could be to use a \ac{cnn} as preprocessing before passing data to the \ac{slam} especially for detecting the bounding boxes of cones in the image data, and estimating the color right from the image data alone before passing it to the slam
\section{Outlook}
Overall this work sets the first step towards driving the Raceyard race cars autonomously in real life, by contributing to the pipeline one of the last essential implementations that is needed before the first autonomous test drive can be commenced. While this thesis by no means provides an implementation that will win races, it provides many theoretical concepts for map generation, ideas for future development along a proof of concept that can very well be used in near future to drive the race car fully autonomously along a track for the very first time in real life.
\todo{0.5p}