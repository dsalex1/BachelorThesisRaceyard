\chapter{Discussion}

\section{Evaluation}
\subsection{Classical Approach}
uncerternity threshhold in covariances auswirkungen analzsieren\\
sicherheit in der karte vs rauschen

zeitlicher verlauf nicht verfolgbar weil
slam ids nicht matchen können beschrieben,
weil particles getrennt kann man nicht gut mathcen,
zu inperformant/wenn springen dann gar nicht

evaluation wie weit voraus notwendign sinnvoll etc
\todo{1p}
\subsubsection{Metric - Deviation From \ac{gt}}
examples for successful failed detection of the centerline
\todo{1p}


\subsection{Machine Learning Approach}

training using mean average loss with ADAM and 
$learning_{rate}=0.001$ proved to be succinct, 15 epochs with 2 batch size, were already enough since the loss dropped quickly and was pretty stable, to prevent overfitting training ended already.\\
low number of training samples already great results\\
different parameters different learning results

\todo{2p}

\subsubsection{Metric - Driving Test}
letting driver test according to algorithm 
\todo{1p}

\section{Comparison of Approaches}
ml more useful in first when there is no map data available, more robust for less accurate map data\\
less plannig ahead possible, work needed to generate map afterwards\\
classical resulting in complete map where planning can be done extensively, but very fragile,
classical only produces completely unusable able when used in the first round with incomplete round, because too incomplete and to noisy data
\todo{1p}
latenz/laufzeit betrachten als metrik

