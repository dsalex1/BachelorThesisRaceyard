\chapter{Introduction}

\section{Motivation}



Autonomous racing sets a competition driven framework for the exploration of autonomous driving
which plays important role in automation of modern transport. One example of such competition is
Formula Student Driverless (FSD).
one team that competes in this counteraction is rosyard, the Formular Student team of the Kiel university

in autonomous racing
cars map generation crucial Problem\\
basis for autonomous driver to decide its path\\
look at different approaches for using data provided by \ac{slam} turn into usable map data for driver
using a classical algorithm and a machine learning approach\\
this thesis systematically compares these 2 approaches for the creation of map in autonomous racing
\todo{0.5p}
\section{Related Work}

einordnung in andere, was soll erforscht werden und oh wunder wird von mir gemacht\\

andere arbeiten nur slam und karte erkennen zusammenm\\
oder neural networks, auf starße aber nicht pylonen eigentlichen\\
aber ich beides
\todo{1p}

\section{Thesis Structure}
in the following chapter basics and technical baground explained\\
in 3rd chapter the details of the classical and \ac{ml} approach, as well as their implementation presented\\
in the 4th chapter the approaches are evaulated and compared,
and in the last chapter the results are summarized and several improvements and ideas for future work are listed
\todo{0.5p}

