\graphicspath{{Chapter/Figs/methods/}}
\chapter{Methods}

\section{Classical Approach}

\subsection{Basis - Master Project by Vaishnav/Agrawal}
based on master project Vaishnav/Agrawal

Schwächen bsheriger Algorithmus, Rot anstatt Gelb, Schwarz nicht klassifizierte Punkte, Grau gänzlich nicht detektierte Punkte,
schwarze Linie ground truth, Der Algorithmus macht die grüne Linie daraus.
wo die Pylonen perfekt erkannt wurden, grüne Linie exakt auf ground truth,
da wo die Farbe nicht erkannt wurde, passieren ganz komische dinge
und da wo oben rechts gar keine Roten erkannt wurden weicht es sehr vom eigentlichen Track ab.
Eingabekarte: https://gyazo.com/645d505ed8a7b1ae5bd6ac8d6a867348
ERgebnis: https://gyazo.com/43237dba0fb02314743c860a9360e520
\todo{0.5p}

\subsection{First Improvement - Guessing Missing Points}
1. readding missing points

erster Ansatz, Punkte auf einer Seite fehlen,
also orthogonal zur Tangente eines Pylons mit dem Abstand der Trackbreite kein andersfarbiger Pylon, dann ergänzt
https://gyazo.com/38577684278cf937f521c3c4cbf3835a
\todo{2p}

\subsection{Second Improvement - Covariance Filtering}

2. ansatz Covarianzen rumgespielt, nur Punkte mit entsprechend kleiner Covarianz
mit covarianz filter: https://gyazo.com/8c58c259adede0fb5b1eb977704b9655
echte daten:
https://gyazo.com/60c4223b8f38d32ed84137b4aa98ae09
\todo{1p}

\section{Machine Learning Approach}
\subsection{Idea and Input/Output Design}
curevature to points 2-8m further down the midline
using immediate environment to predict the curvature of the line segment current on
\todo{1p}
\subsection{Modelling as Image Regressing Problem Using an CNN}
mapping of input values to have a flatter distribution of values
\todo{3p}