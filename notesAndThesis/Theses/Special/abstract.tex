% ************************** Thesis Abstract *****************************
% Use `abstract' as an option in the document class to print only the titlepage and the abstract.
\begin{abstract}

Automation is the natural direction to take on in the seek of increased safety, efficiency and passenger comfort, wherein autonomous racing sets a competition-driven framework for the exploration of new ideas. The problem of autonomous racing can be split into three main parts: landmark detection and tracking, map generation and trajectory planning, and controlling the vehicle. Map generation is a primary building block in which landmarks in a virtual space provided by a SLAM algorithm are used to create a map that can be used to determine the neccessary driving parameters. Two solutions for this problem are presented and compared in this work: an extension of a previously used classical algorithm by Vaishnav/Agrawal and a novel machine learning based algorithm. Three improvements to the classical algorithm are proposed: an improved spatial ordering, the readdition of missing points using heuristic guessing, and a filtering method based on the certainty of the detection. Even with these improvements, it is shown that the algorithm is too brittle to produce accurate results with erroneous input data. The machine learning algorithm is very error-resilient while still approximating sufficiently enough to be used in a simulated environment. Also, the runtime of these algorithms is shown to differ by an order of magnitude.
\end{abstract}
